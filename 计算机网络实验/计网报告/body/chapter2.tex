\section{心得体会与建议}
\subsection{心得体会}
\hspace*{2em}在 Socket 编程实验中,通过自己编写代码真正实现服务器端处理客户端的 HTTP 请求,学到了很多,也和我们日常上网的体验相联系起来。\\
\hspace*{2em}在上课的过程中,我们学习了关于 Socket 以及 HTTP 协议的理论知识,了解了理论知识在一定程度上也对网络请求以及响应有了了解,但是刚开始做实验的时候还是有一种很难着手去做的感觉。在明白了是以自己设备的文件系统作为服务器,并用自己的浏览器作为客户端去进行访问,而访问的IP就是本机IP之后,对整个实验有了一个框架性的理解,至少可以开始着手写了。\\
\hspace*{2em}后来看了老师给的文档,对于 Windows 平台下网络通信协议的开发接口 Winsock 的API有了大致的了解,大概知道整个系统应该从哪开始和怎么构建。接下来难题变成了自己怎么去解析请求报文。一开始我打算自己根据 HTTP 请求头来进行解析,但是失败了。然后找到了正则表达式的解析方式,这就十分轻松简易了。\\
\hspace*{2em}第二次和第三次的实验相比起来没有第一次那样无从下手的感觉。第二次的可靠传输协议在老师给的停等协议的基础上,理解了GBN、SR和TCP的原理就可以模仿着写出来了。第三次实验图形化的界面有一种做 Logisim 实验的感觉,把路由器、交换机等等硬件模拟出来,那些看不到的信息配置也通过图形化界面让我们可以直接输入接触到,感觉很有意思。\\
\subsection{建议}
\hspace*{2em}在实验中都给了很详细的 API 说明文档,这对写代码很有帮助,这样我们就不用上网去找查 API 了。但是如果能把这个实验总体要达到的效果告诉我们,比如实验一中要用浏览器模拟客户端,我们对实验有一个大体的认识就不会产生无从下手的感觉了。\\
\hspace*{2em}总体而言,像实验一和实验三这样贴近生活的实验,做起来还是很有意思的。\\